\documentclass[11pt,fleqn]{scrartcl}

\usepackage[latin1]{inputenc}
\usepackage{ngerman}
\usepackage{amsmath,amssymb,amstext}
\usepackage{graphicx}
\usepackage[automark]{scrpage2}

\title{Web Service Hilfe}
\author{Christoph Hartmann, Michael Perscheid}
\date{\today{} in Potsdam}

\pagestyle{scrheadings}

\ifoot[Christoph Hartmann, Michael Perscheid]{Christoph Hartmann, Michael Perscheid}
\cfoot[]{}
\ofoot[\thepage]{\thepage}

\begin{document}

\maketitle

\quote {Bei diesem Dokument handelt es sich um eine Hilfestellung zur Beschreibung einer eigenen Hilfeseite zur Nutzung von Web Service Security mittels des tele-TASK-Frameworks.}

\tableofcontents

\section{Nutzungsbedingungen}
\label{wsh:Nutzungsbedingungen}
Hier bitte die Nutzungsbedingungen f�r die Verwendung von Web Services des tele-TASK-Frameworks einf�gen.

\subsection{Erstellen eines Accounts}
\label{wsh:ErstellenEinesAccounts}
Wie und wo erstelle ich einen Account oder bekomme ihn her?
Ein Account muss vorher in der verwendeten Benutzerverwaltung angelegt werden. Auch das Passwort muss dort bekannt sein. 

\subsection{Gastaccount}
\label{wsh:Gastaccount}
Was versteht der Benutzer unter einem Gastaccount?

\section{Authentifizierung - Username Token Profil}
\label{wsh:Authentifizierung}
Zur Absicherung der Web Services gegen nicht authorisierte Personen wurde das Username Token Profil in PHP implementiert. Dabei handelt sich um einen Teil des Standards Web Service Security von OASIS\footnote{www.oasis-open.org}. Hierbei stehen dem Benutzer zwei M�glichkeiten zur Verf�gung sich zu identifizieren:

\subsection{Passwort Klartext}
\label{wsh:PasswortKlartext}
Die einfachste und unsicherste Methode sich beim Server zu authentifizieren, besteht darin nur seinen Accountnamen und Passwort im Header einer jeden Soap Message zu �bermitteln. Dabei sind alle Daten unverschl�sselt und Replay-Attacken\footnote{Dabei wird die Soap Message abgefangen und beliebig oft erneut gesendet.} sind leicht ausf�hrbar. Daher sollte diese Variante vermieden und die M�glichkeit des verschl�sselten Passwortes genutzt werden.

\subsection{Verschl�sseltes Passwort}
\label{wsh:PasswortDigest}
Verschl�sselte Passw�rter\footnote{PasswortDigest} werden mittels einem Zufallswert und einem Zeitstempel abgesichert. Diese Daten werden neben dem Accountnamen auch im Header der Nachricht mitgef�hrt. Dementsprechend ist jede Anfrage nur noch genau einmal g�ltig, da sonst der Zufallswert sich auf Serverseite doppelt oder die Anfrage zu alt ist.

\subsection{MD5 Passw�rter}
\label{wsh:MD5Passwort}
Da einerseits auf Serverseite aus Sicherheitsgr�nden nur die MD5-Hashwerte der Passw�rter zur Verf�gung stehen und auf der anderen Seite klare Passw�rter nicht zu empfehlen sind, \textbf{gilt, dass alle Anfragen mit dem Passwort�quivalent, sprich ihrem MD5-Hashwert, gesendet werden m�ssen.} Dieser Punkt kann bei Web Service Anfragen schnell zu einem vermeidbaren Problem werden.


\section{Client Implementierungen}
\label{wsh:ClientImplementation}
Zur Benutzung der Web Services empfiehlt es sich einen Client zu benutzen. Dabei werden hier drei in den gebr�uchlichsten Sprachen vorgestellt. Sie wurden alle in Testf�llen erfolgreich �berpr�ft. Bei der Verwendung eines anderen Clients, muss auf die Kompatibilit�t zu Web Service Security 1.0 geachtet werden.

\subsection{PHP Client}
\label{wsh:PHPClient}
Der PHP Client\footnote{URL zur client.php} ist ebenso wie der Server eine Eigenentwicklung. Er ist auf Grund von PHP Eigenheiten nicht interoperabel. Jedoch funktioniert er sehr gut mit den tele-TASK Web Services und kann hier uneingeschr�nkt verwendet werden. Dabei wird ausschliesslich Passwort Digest verwendet und der Anwender des Clients muss nur in einem erweiterten Konstruktor Name und Passwort angeben. Alles weitere, wie z.B. Header erzeugen, ist vom Client gekapselt und f�r den Anwender nicht zu beachten.

\subsection{C\# Client}
\label{wsh:CSharpClient}
F�r C\# wird die von Microsoft implementierte Version der Web Services Enhancements 3.0 empfohlen.\footnote{Microsoft Web Services Enhancements 3.0 www.microsoft.com} In dieser Erweiterung ist auch das Username Token Profile vorhanden, welches genutzt werden muss, um eine Verbindung mit tele-TASK Web Services herstellen zu k�nnen.

\subsection{Java Client}
\label{wsh:JavaClient}
In Java wurde erfolgreich eine Verbindung mittels Java Axis hergestellt.\footnote{Java Axis 1.3 http://ws.apache.org/axis/} Diese Erweiterung bietet, analog zu Microsoft, viele M�glichkeiten Web Services zu nutzen, darunter ist auch das Username Token Profile.

\section{Liste der verf�gbaren Web Services}
\label{wsh:ListeWebServices}
Liste der verf�gbaren Web Service hier einf�gen.


%\section{Allgemeine Fragen zum Server}
%\label{wsh:AllgemeineFragen}
%\subsection{Wie kann ich einen Server aufsetzen?}
%\label{wsh:ServerAufsetzen}

%Um einen Server aufzusetzen, muss als erstes ein Service implementiert werden. Diese PHP Klasse muss mittels PHP 5 Features objektorientiert implementiert werden. Anschlie�end wird eine WSDL-Beschreibung erstellt. Dazu kann entweder der Generator benutzt werden oder die Beschreibung wird per Hand erstellt. Die erste Variante ist aus Kompatibilit�tsgr�nden dringend zu empfehlen.
%Anschlie�end muss der Server aufgesetzt werden. Dies ist mit folgendem CodeSchnipsel m�glich:
%<addCodehere>

%Der Ablauf l�sst sich wie folgt zusammenfassen:
%\begin{itemize}
%	\item PHP Klasse erstellen
%	\item WSDL genierieren
%	\item Server mit WSDL aufsetzen, Klasse als Implementierung angeben
%\end{itemize}
 

%\subsection{Wie kann ich Benutzer meiner Web Services authentifizieren lassen?}
%\label{wsh:BenutzerAuthentifizieren}

%Als erstes muss eine Benutzerverwaltung vorhanden sein. Sie muss mindestes ein Passwort, welches sich md5 gehasht auslesen l�sst, zu einem Benutzernamen speichern. Mittels dieser Benutzerverwaltung kann sp�ter ein Vergleich erstellt werden. 
%Die Anbindung an den Server erfolgt �ber eine ge�nderte Implementierung des ICheckUserRunnable-Interface. Da m�ssen die Methoden getPassword, setAccount, setGuestAccout neu implementiert werden. Der Rest ist im ExtendedSoapServer schon implementiert. Die neue ICheckUserRunnable an den XmlSoapSecParser binden. 
%Nachdem der XmlParser an den ExtendedSoapServer gebunden wurde, kann die Authentifizierung verwendet werden.

%Kurz Zusammengefasst:
%\begin{itemize}
%	\item Aufsetzen der Benutzerverwaltung
%	\item Anpassen des ICheckUserRunnable an die Rechteverwaltung
%	\item Benutzer Username und Passwort mitteilen
%  \item Anstatt des SoapServers den Extended Soap Server verwenden
%  \item XML Parser binden
%\end{itemize}

%\subsection{Implementieren einer eigenen Sicherheitsl�sung}
%\label{wsh:SicherheitImplementierung}
%Das implementieren von weiteren Sicherheitsaspekten kann mittels eines neuen XML Parser erfolgen. Dieser muss dann an den ExtendedSoapServer gebunden werden.

%Es muss daher folgendes erledigt werden:
%\begin{itemize}
%	\item Schreiben eines XML Parser, welcher von ExtendedXmlParser erbt
%	\item Binden an den ExtendedSoapServer
%\end{itemize}

\end{document}
